\documentclass{article}

\usepackage[ngerman]{babel}
\usepackage[utf8]{inputenc}

\usepackage{amsmath}
\usepackage{amsthm}
\usepackage{amssymb}
\usepackage{tikz}
\usepackage{bm}

\usepackage{cite}
\usepackage{graphicx}
\graphicspath{ {./images/} }


\newcommand{\N}{\ensuremath{\mathbb{N}}}
\newcommand{\Z}{\ensuremath{\mathbb{Z}}}
\newcommand{\Q}{\ensuremath{\mathbb{Q}}}
\newcommand{\R}{\ensuremath{\mathbb{R}}}
\newcommand{\E}{\ensuremath{\mathbf{E}}}
\newcommand{\X}{\ensuremath{\mathcal{X}}}
\newcommand{\Prob}{\ensuremath{\mathbf{P}}}
\newcommand{\Var}{\ensuremath{\mathrm{Var}}}
\newcommand{\Cov}{\ensuremath{\mathrm{Cov}}}
\newcommand{\TV}{\ensuremath{\mathrm{TV}}}
\newcommand{\one}{\ensuremath{\bm{1}}}
\newcommand{\A}{\ensuremath{\mathcal{A}}}

\newtheorem{theorem}{Theorem}[section]
\newtheorem{satz}{Satz}[section]
\newtheorem{lemma}[satz]{Lemma}
\newtheorem{korollar}[satz]{Korollar}
\theoremstyle{definition}
\newtheorem{definition}[satz]{Definition}
\newtheorem{bemerkung}[satz]{Bemerkung}
\newtheorem{beispiel}[satz]{Beispiel}


\makeatletter
 \def\blfootnote{\xdef\@thefnmark{}\@footnotetext}

\newcommand{\handouttitle}[4]
 {\begin{center}
    \Large #4
  \end{center}

  \bigskip

  \noindent
  #1 (\textsf{#2})
  \hfill
  #3
  \blfootnote{Seminar \glqq Markovketten und stochastische Algorithmen\grqq, SS 2020, Universität Bonn}

  \noindent
  \rule{\linewidth}{.5pt}
  
  \bigskip
  \@afterindentfalse\@afterheading
 }

\makeatother

\begin{document}

  \handouttitle{Ben Breitinger}
               {ben.breitinger@uni-bonn.de}
               {30. Juni 2020}
               {Untere Schranken für Mischzeiten, Bottlenecks}
  
  \begin{abstract}
    Da wir das Konvergenzverhalten nicht nur von oben, sondern auch von unten beschränken wollen, interessieren uns untere Schranken für die Mischzeit. 
    Dazu führen wir das Flaschenhals-Verhältnis, welches ein auf Graphen besonders anschauliches Konstrukt ist, ein, womit wir Abschätzungen für untere Schranken von $t_{\text{mix}}$ erhalten werden. 
    Diese wenden wir auf zwei aneinandergeklebte Tori und einen Binärbaum an.

    In einem weiteren Abschnitt werden wir eine statistische Methode entwickeln, um untere Schranken für $t_{\text{mix}}$ zu erhalten. Mithilfe dieser neugewonnen Techniken werden wir eine Abschätzung für die Mischzeit eines Random Walks auf dem Hyperwürfel herleiten. 
  \end{abstract}
\end{document}
