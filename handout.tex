\documentclass{article}

\usepackage[ngerman]{babel}
\usepackage[utf8]{inputenc}

\usepackage{amsmath}
\usepackage{amsthm}
\usepackage{amssymb}
\usepackage{mathrsfs}
\usepackage{bm}

\usepackage{cite}
\usepackage{graphicx}
\graphicspath{ {./images/} }


\newcommand{\N}{\ensuremath{\mathbb{N}}}
\newcommand{\Z}{\ensuremath{\mathbb{Z}}}
\newcommand{\Q}{\ensuremath{\mathbb{Q}}}
\newcommand{\R}{\ensuremath{\mathbb{R}}}
\newcommand{\G}{\ensuremath{\mathscr{G}}}
\newcommand{\w}{\ensuremath{\omega}}
\newcommand{\W}{\ensuremath{\Omega}}
\newcommand{\F}{\ensuremath{\mathscr{F}}}

\newcommand{\stcomp}[1]{{#1}^{\mathsf{c}}}





\newtheorem{theorem}{Theorem}[section]
\newtheorem{satz}{Satz}[section]
\newtheorem{lemma}[satz]{Lemma}
\newtheorem{korollar}[satz]{Korollar}
\theoremstyle{definition}
\newtheorem{definition}[satz]{Definition}
\newtheorem{bemerkung}[satz]{Bemerkung}
\newtheorem{beispiel}[satz]{Beispiel}


\makeatletter
 \def\blfootnote{\xdef\@thefnmark{}\@footnotetext}

\newcommand{\handouttitle}[4]
 {\begin{center}
    \Large #4
  \end{center}

  \bigskip

  \noindent
  #1 (\textsf{#2})
  \hfill
  #3
  \blfootnote{Seminar \glqq Statistische Mechanik von Gittersystemen\grqq, SoSe 2021, Universität Bonn}

  \noindent
  \rule{\linewidth}{.5pt}
  
  \bigskip
  \@afterindentfalse\@afterheading
 }

\makeatother

\begin{document}

  \handouttitle{Ben Breitinger}
               {ben.breitinger@uni-bonn.de}
               {DATUM}
               {Infinite-Volume Gibbs Measures II}
  
  \begin{abstract}
    ABSTRACT
  \end{abstract}
  
  \section{Einleitung}
  In der vorigen Woche haben wir gesehen, dass $\G(\pi) \neq \emptyset$, falls $\pi = \{ \pi_{\Lambda} \}
  _{\Lambda \Subset \Z^d}$ eine quasilokale Spezifikation ist. Nun wollen wir uns der Frage widmen, wann ein $\mu \in \G(\pi)$ eindeutig ist, und welche Bedingungen wir dafür an $\pi$ stellen müssen. Diese Frage steht in direktem Zusammenhang mit der Frage nach Phasenübergängen. 
  
  Wir betrachten zwei zentrale Methoden - den \textit{Satz von Dobrushin} und die \textit{Cluster Expansion} -, welche uns Kriterien an die Hand geben werden, mit denen wir einige Anwendungen betrachten werden können.
  
  Hat die Spezifikation $\pi$ gewisse Symmetrieeigenschaften, so können wir auch auf Symmetrie der Gibbs-Maße schließen. Folglich werden wir Symmetrien und Translationen mitsamt ihren Anwendungen betrachten.
  
  Am Ende möchte ich noch einen kleinen Exkurs in einen interessanten Zusammenhang zwischen Gibbs-Maßen und einer physikalischen Kenngröße mithilfe des sogenannten Variationsprinzip herstellen.
  
  \section{Eindeutigkeit von Gibbs-Maßen}
  \subsection{Der Satz von Dobrushin}
  
  Der Satz von Dobrushin wird uns bei geringen \glqq Fluktuationen \grqq{} Eindeutigkeit des Gibbs-Maßes liefern. Um dies etwas genauer zu fassen, führen wir zunächst einiges an Notation ein.
  Wir betrachten in dieser Sektion sehr häufig stochastische Kerne, die auf nur einem Gitterpunkt $i \in \Z^d$ mit der Randbedingung $\w$ messen. Deshalb schreibe ich $\pi_i (\cdot \mid \w) := \pi_{\{i\}}( \cdot \mid \w)$. Zunächst erinnern wir uns an folgende Definiton:
  \begin{definition}
   Seinen $\mu$ und $\nu$ Wahrscheinlichkeitsmaße auf dem Messraum $(\W, \F)$. Die \emph{Variationsdistanz} ist definiert als 
   \[
    \| \mu - \nu \|_{TV} := 2\sup_{A \in \F} | \mu(A) - \nu(A) |.
   \]
  \end{definition}
  Im Falle, dass $\W$ abzählbar ist, gilt auch $\| \mu - \nu \|_{TV} = \sum_{\w \in \W} | \mu(w) - \nu(w) |$. Den Beweis dafür verschieben wir in den Anhang \ref{}. Angewandt auf unsere Situation ergibt sich
  \[
    \| \pi_i(\cdot \mid \w) - \pi(\cdot \mid \w^{\prime}) \|_{TV} = \sum_{\eta_i = \pm 1} \left| \pi_i(\eta_i \mid \w) - \pi_i(\eta_i \mid \w^{\prime}) \right|.
   \]
  
  Um den Satz von Dobrushin formulieren zu können, benötigen wir außerdem die Bezeichnungsweisen
  \[
   c_{ij}(\pi) := \sup_{\substack{\w, \w^{\prime} \in \W \\ \w_k=\eta_k \forall k \neq i}} \| \pi_i(\cdot \mid \w) - \pi_i (\cdot \mid \w^{prime}) \| \quad \text{und} \quad c(\pi) := \sup_{i \in \Z^d} \sum_{j \in \Z^d}c_{ij}(\pi).
  \]

  Wir können $c(\pi)$ als eine Kenngröße verstehen, die beschreibt, wie groß der Einfluss eines Gitterpunktes $i \in \Z^d$ auf alle anderen Gitterpunkte unter $\pi$ höchstens sein kann. Damit können wir nun den Satz von Dobrushin formulieren.
  
  \begin{satz}
   Sei $\pi$ eine quasilokale Spezifikation, die 
   \begin{equation}
    c(\pi) < 1
   \end{equation}
   erfüllt. Dann ist das durch $\pi$ spezifizierte Wahrscheinlichkeitsmaß eindeutig, also $|\G(\pi)| = 1$.
  \end{satz}

  Bevor wir uns dem Beweis zuwenden, wollen wir zunächst etwas Notation einführen.
  \begin{definition}
   Definiere für eine Funktion $f: \W \to \R$ durch
   \[
    \delta_i(f) := \sup_{\substack{\w, \w^{\prime} \in \W \\ \w_k=\eta_k \forall k \neq i}} | f(\w) - f(\eta) |
   \]
   die Oszillation von $f$ in $i \in \Z^d$. Bezeichne mit
   \[
    \Delta(f) = \sum_{i \in \Z^d} \delta_i(f)
   \]
   die Gesamtoszillation. Wir bezeichnen die Menge der Funktionen, für die $\Delta(f) < \infty$ gilt, mit $\mathscr{O(\W)}$ und $C_{\mathscr{O}}(\W) = \mathscr{O}(\W) \cap C(\W)$. 
  \end{definition}
  
  Mithilfe der Dreiecksungleichung ist leicht einzusehen, dass für zwei Konfigurationen $\w, \eta \in \W$ mit $\w_{\stcomp{\Lambda}} = \eta_{\stcomp{\Lambda}}$ 
  \[
   | f(\w) - f(\eta) | \leq \sum_{i \in \Lambda} \delta_i(f)
  \]
  gilt.

  Ganz ähnlich wird das folgende Lemma eingesehen.
  \begin{lemma}
   Sei $f \in C_{\mathscr{O}}(\W)$. Dann gilt $\Delta(f) \geq \sup f - \inf f$.
  \end{lemma}
  \begin{proof}
   Da $f$ stetig ist, nimmt es sein Maximum und sein Minimum an. S
  \end{proof}

  
  
  \subsubsection{Anwendung auf Gibbs-Spezifikationen}
  
  \subsection{Eine alternative Charakterisierung der Eindeutigkeit und Cluster Expansion}
  \section{Symmetrien und Translationen}



\end{document}
